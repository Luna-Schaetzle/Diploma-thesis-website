\documentclass{beamer}
\usetheme{Madrid}

\title{Luminara Schüler-KI-Plattform}
\subtitle{Projektvorstellung}
\author{Luna Schätzle}
\date{\today}

\begin{document}

% Titelseite
\begin{frame}
    \titlepage
\end{frame}

% Folie: Einleitung
\begin{frame}{Einleitung}
    \begin{itemize}
        \item Ziel: Unterstützung von Schüler durch Interaktion mit Künstlicher Intelligenz (KI).
        \item Plattform kombiniert:
        \begin{itemize}
            \item Nutzung der ChatGPT-API.
            \item Lokale KI-Modelle wie Ollama auf Schulservern.
        \end{itemize}
        \item Fokus: Vielseitige Lernunterstützung und Aufgabenbearbeitung.
    \end{itemize}
\end{frame}

% Folie: Funktionen
\begin{frame}{Funktionen der Plattform}
    \begin{itemize}
        \item \textbf{Programmierbot}: Hilfe beim Programmieren mit Codebeispielen.
        \item \textbf{OCR (Bild zu Mitschrift)}: Konvertiert Bilder in Text.
        \item \textbf{Allgemeiner Chatbot}: Angepasster Chat für allgemeine Anfragen.
        \item \textbf{Bilderkennung}: Analysiert hochgeladene Bilder.
        \item \textbf{Mathe-Bot}: Unterstützung bei mathematischen Problemen.
        \item \textbf{Testvorbereitung}: Generiert Fragen zur Testvorbereitung.
        \item \textbf{Chatten mit API}: Erweiterte Konversationen mit ChatGPT und Claude AI.
    \end{itemize}
\end{frame}

% Folie: Konto-Modell
\begin{frame}{Konto-Modell}
    \textbf{Benutzerrollen:}
    \begin{itemize}
        \item \textbf{Student-Accounts:} Monatliche Token für Premium-KI-Modelle. Nicht genutzte Token werden übertragen.
        \item \textbf{Student+ Accounts:} Erweitertes Token-Kontingent für intensivere Nutzung.
        \item \textbf{Admin- und Dev-Accounts:} Unbegrenzte Token, Verwaltungs- und Entwicklungsrechte.
    \end{itemize}
    \textbf{Ziel:} Effiziente und faire Ressourcennutzung für alle Benutzergruppen.
\end{frame}

% Folie: Technologie-Stack
\begin{frame}{Technologie-Stack}
    \textbf{Frontend:} Vue.js \\
    \textbf{Backend:} Firebase \\
    \textbf{KI-Modelle:}
    \begin{itemize}
        \item ChatGPT API
        \item Ollama (lokale Modelle)
        \item Claude AI
    \end{itemize}
    \textbf{Weitere Tools:}
    \begin{itemize}
        \item OCR-Tools (z. B. Tesseract)
        \item Bildverarbeitung (z. B. TensorFlow, OpenCV)
        \item Datenbanken (z. B. MongoDB, PostgreSQL)
    \end{itemize}
\end{frame}

% Folie: Nutzung
\begin{frame}{Nutzung der Plattform}
    \begin{enumerate}
        \item \textbf{Registrierung und Anmeldung:} Schüler melden sich an.
        \item \textbf{Funktionsauswahl:} Nutzer wählen aus Funktionen wie OCR, Mathe-Bot, Bilderkennung, etc.
        \item \textbf{Interaktion mit KI:} Aufgaben hochladen, Fragen stellen, interaktive Materialien nutzen.
    \end{enumerate}
\end{frame}

% Folie: Lizenz und Kontakt
\begin{frame}{Lizenz und Kontakt}
    \textbf{Lizenz:} GNU General Public License v3.0 \\
    \vspace{1cm}
    \textbf{Kontakt:}
    \begin{itemize}
        \item E-Mail: \texttt{luminaraai.website@gmail.com}
        \item GitHub: \texttt{Luna-Schaetzle}
    \end{itemize}
\end{frame}

% Abschlussfolie
\begin{frame}
    \centering
    \textbf{Vielen Dank für Ihre Aufmerksamkeit!}
\end{frame}

\end{document}
